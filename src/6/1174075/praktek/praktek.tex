\subsection{Praktek}
\subsubsection{Tugas No 1}
\hfill \break
Dibawah ini merupakan penggunaan subplot dan plot bar
\lstinputlisting[firstline=8, lastline=28]{src/6/1174075/praktek/1174075_bar.py}
Dan dibawah ini merupakan cara pemangilannya
\lstinputlisting[firstline=8, lastline=8]{src/6/1174075/praktek/main_sekar.py}
\lstinputlisting[firstline=13, lastline=13]{src/6/1174075/praktek/main_sekar.py}

\subsubsection{Tugas No 2}

\hfill \break

Dibawah ini merupakan penggunaan subplot dan plot scatter
\lstinputlisting[firstline=8, lastline=28]{src/6/1174075/praktek/1174075_scatter.py}
Dan dibawah ini merupakan cara pemangilannya
\lstinputlisting[firstline=9, lastline=9]{src/6/1174075/praktek/main_harun.py}
\lstinputlisting[firstline=14, lastline=14]{src/6/1174075/praktek/main_sekar.py}

\subsubsection{Tugas No 3}

\hfill \break

Dibawah ini merupakan penggunaan subplot dan plot pie
\lstinputlisting[firstline=8, lastline=50]{src/6/1174075/praktek/1174075_pie.py}
Dan dibawah ini merupakan cara pemangilannya
\lstinputlisting[firstline=10, lastline=10]{src/6/1174075/praktek/main_sekar.py}
\lstinputlisting[firstline=15, lastline=15]{src/6/1174075/praktek/main_sekar.py}

\subsubsection{Tugas No 4}

\hfill \break

Dibawah ini merupakan penggunaan subplot dan plot bar
\lstinputlisting[firstline=8, lastline=28]{src/6/1174075/praktek/1174075_plot.py}
Dan dibawah ini merupakan cara pemangilannya
\lstinputlisting[firstline=11, lastline=11]{src/6/1174075/praktek/main_sekar.py}
\lstinputlisting[firstline=16, lastline=16]{src/6/1174075/praktek/main_sekar.py}

\subsection{Penanggan Error}

\hfill \break

Berikut ini merupakan cara penangganan errornya
\lstinputlisting[firstline=8, lastline=14]{src/6/1174075/praktek/1174075.py}