\section{Kaka Kamaludin}
\subsection{Soal 1}
\lstinputlisting[firstline=1, lastline=8]{src/4/1174067/Praktek/1174067_csv.py}
\subsection{Soal 2}
\lstinputlisting[firstline=8, lastline=14]{src/4/1174067/Praktek/1174067_csv.py}
\subsection{Soal 3}
\lstinputlisting[firstline=1, lastline=6]{src/4/1174067/Praktek/1174067_pandas.py}
\subsection{Soal 4}
\lstinputlisting[firstline=6, lastline=11]{src/4/1174067/Praktek/1174067_pandas.py}
\subsection{Soal 5}
\lstinputlisting[firstline=11, lastline=16]{src/4/1174067/Praktek/1174067_pandas.py}
\subsection{Soal 6}
\lstinputlisting[firstline=16, lastline=21]{src/4/1174067/Praktek/1174067_pandas.py}
\subsection{Soal 7}
\lstinputlisting[firstline=21, lastline=27]{src/4/1174067/Praktek/1174067_pandas.py}
\subsection{Soal 8}
\lstinputlisting[firstline=1, lastline=17]{src/4/1174067/Praktek/main.py}
\subsection{Soal 9}
\lstinputlisting[firstline=1, lastline=29]{src/4/1174067/Praktek/main2.py}
\subsection{keterampilan Penanganan Error}

SyntaxError: invalid token

salah dalam penulisan " import 1174067\textunderscore csv ", seharusnya "pkg = \textunderscore \textunderscore import\textunderscore \textunderscore('1174067\textunderscore csv')"

%%%%%%%%%%%%%%%%%%%%%%%%%%%%%%%%%%%%%%%%%%%%%%%%%%%%%%%%%%%%%%%%%%%%%%%%%%%%%%%%%%%%%%%%%%%%%%%%%%%%%%%%%%%%%%%%%

\section{Alfadian Owen}
\subsection{Soal 1}
Buatlah fungsi untuk membuka file csv dengan lib csv mode list
\lstinputlisting[firstline=8, lastline=14]{src/4/1174091/praktek/1174091_csv.py}

\subsection{Soal 2}
Buatlah fungsi untuk membuka file csv dengan lib csv mode dictionary
\lstinputlisting[firstline=16, lastline=21]{src/4/1174091/praktek/1174091_csv.py}

\subsection{soal 3}
Buatlah fungsi  untuk membuka csv dengan lib pandas mode list
\lstinputlisting[firstline=7, lastline=11]{src/4/1174091/praktek/1174091_pandas.py}

\subsection{Soal 4}
Buatlah fungsi untuk membuka file csv dengan lib pandas mode dictionary
\lstinputlisting[firstline=34, lastline=38]{src/4/1174091/praktek/1174091_pandas.py}

\subsection{soal 5}
Buat fungsi baru di NPM pandas.py untuk mengubah format tanggal menjadi standar dataframe
\lstinputlisting[firstline=19, lastline=22]{src/4/1174091/praktek/1174091_pandas.py}

\subsection{soal 6}
Buat fungsi baru di NPM pandas.py untuk mengubah index kolom
\lstinputlisting[firstline=24, lastline=27]{src/4/1174091/praktek/1174091_pandas.py}

\subsection{soal 7}
Buat fungsi baru di NPM pandas.py untuk mengubah atribut atau nama kolom
\lstinputlisting[firstline=29, lastline=32]{src/4/1174091/praktek/1174091_pandas.py}

\subsection{Soal 8}
Buat program main yang menggunakan library NPM csv yang membuat dan membaca file csv
\lstinputlisting[firstline=8, lastline=11]{src/4/1174091/praktek/main.py}

\subsection{Soal 9}
Buat program main2.py yang menggunakan library NPM pandas.py yang membuat dan membaca file csv
\lstinputlisting[firstline=8, lastline=20]{src/4/1174091/praktek/main.py}


%%%%%%%%%%%%%%%%%%%%%%%%%%%%%%%%%%%%%

