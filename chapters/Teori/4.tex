\section{Kaka Kamaludin}
\subsection{Soal 1}
CSV (comma separated values)

seperti namanya CSV, merupakan file yang berisi data berupa angka dan teks, di setiap data atau nilai dipisahkan dengan tanda koma (,) dan data tersebut ditampilkan sebagai tabel. file csv bisa dibuka menggunakan teks editor apapun, selain itu csv juga bisa dibuka menggunakan excel. file csv berfungsi untuk menyimpan data dalam bentuk teks yang nantinya digunakan untuk keperluan tertentu.

contoh file employee\textunderscore birthday.csv berisi:

name,department,birthday month
John Smith,Accounting,November
Erica Meyers,IT,March

\subsection{Soal 2}
semua text editor, Excel, tinggal save as *.csv

\subsection{Soal 3}
bagaimana cara menulis dan membaca file csv di excel atau spreadsheet

Cara menulis:
\begin{itemize}
	\item ketik saja data yang anda butuhkan
	\item save as *.csv
\end{itemize}

Cara membaca:
\begin{itemize}
	\item pilih file *.csv
	\item open with exel/spreadsheet
\end{itemize}

\subsection{Soal 4}
sejarah library csv

CSV merupakan format yang paling standar untuk import dan export database ataupun spreadsheet. Format CSV digunakan selama bertahun-tahun sebelum upaya untuk menggambarkan format dengan cara standar di RFC 4180. 

\subsection{Soal 5}
sejarah library pendas

pandas merupkan library open source berlisensi BSD dan pandas merupakan proyek yang disponsori oleh NumFOCUS, menyediaka kinerja tinggi, struktur data yang mudah digunakan dan tools analisis untuk bahasa pemrograman python.  

\subsection{Soal 6}
fungsi-fungsi yang terdapat di library csv
\begin{itemize}
	\item csv.reader
	
	membaca file csv file, kolom pertama berurutan dengan nomor row. 
	
	\item csv.DictReader
	
	
	membaca file csv file,key berurutan dengan row sesuai kolom pertama.
		
	\item csv.writer
	
	membuka file csv yang sudah di deklarasi dan menulisnya kedalam file yang dibuat tadi.
		
	\item csv.DictWriter
	
	membuka file csv yang sudah di deklarasi dan menulisnya kedalam file yang dibuat tadi.	
	
\end{itemize}

\subsection{Soal 7}
fungsi-fungsi yang terdapat di library csv
\begin{itemize}

	\item pandas.read\textunderscore csv

	membaca file csv dan menampilkannya sebagai dataframe.
	
\end{itemize}

%%%%%%%%%%%%%%%%%%%%%%%%%%%%%%%%%%%%%%%%%%%%%%%%%%%%%%%%%%%%%%%%%%%%%%%%%%%%%%%%%%%%%%%%%%%%%%%%%%%