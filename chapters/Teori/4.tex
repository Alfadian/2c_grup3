\section{Kaka Kamaludin}
\subsection{Soal 1}
CSV (comma separated values)

seperti namanya CSV, merupakan file yang berisi data berupa angka dan teks, di setiap data atau nilai dipisahkan dengan tanda koma (,) dan data tersebut ditampilkan sebagai tabel. file csv bisa dibuka menggunakan teks editor apapun, selain itu csv juga bisa dibuka menggunakan excel. file csv berfungsi untuk menyimpan data dalam bentuk teks yang nantinya digunakan untuk keperluan tertentu.

contoh file employee\textunderscore birthday.csv berisi:

name,department,birthday month
John Smith,Accounting,November
Erica Meyers,IT,March

\subsection{Soal 2}
semua text editor, Excel, tinggal save as *.csv

\subsection{Soal 3}
bagaimana cara menulis dan membaca file csv di excel atau spreadsheet

Cara menulis:
\begin{itemize}
	\item ketik saja data yang anda butuhkan
	\item save as *.csv
\end{itemize}

Cara membaca:
\begin{itemize}
	\item pilih file *.csv
	\item open with exel/spreadsheet
\end{itemize}

\subsection{Soal 4}
sejarah library csv

CSV merupakan format yang paling standar untuk import dan export database ataupun spreadsheet. Format CSV digunakan selama bertahun-tahun sebelum upaya untuk menggambarkan format dengan cara standar di RFC 4180. 

\subsection{Soal 5}
sejarah library pendas

pandas merupkan library open source berlisensi BSD dan pandas merupakan proyek yang disponsori oleh NumFOCUS, menyediaka kinerja tinggi, struktur data yang mudah digunakan dan tools analisis untuk bahasa pemrograman python.  

\subsection{Soal 6}
fungsi-fungsi yang terdapat di library csv
\begin{itemize}
	\item csv.reader
	
	membaca file csv file, kolom pertama berurutan dengan nomor row. 
	
	\item csv.DictReader
	
	
	membaca file csv file,key berurutan dengan row sesuai kolom pertama.
		
	\item csv.writer
	
	membuka file csv yang sudah di deklarasi dan menulisnya kedalam file yang dibuat tadi.
		
	\item csv.DictWriter
	
	membuka file csv yang sudah di deklarasi dan menulisnya kedalam file yang dibuat tadi.	
	
\end{itemize}

\subsection{Soal 7}
fungsi-fungsi yang terdapat di library csv
\begin{itemize}

	\item pandas.read\textunderscore csv

	membaca file csv dan menampilkannya sebagai dataframe.
	
\end{itemize}

%%%%%%%%%%%%%%%%%%%%%%%%%%%%%%%%%%%%%%%%%%%%%%%%%%%%%%%%%%%%%%%%%%%%%%%%%%%%%%%%%%%%%%%%%%%%%%%%%%%

\section{Alfadian Owen}
\subsection{Pemahaman Teori}
\begin{enumerate}
\item Apa itu fungsi file csv? jelaskan sejarah dan contoh.

CSV adalah tipe file khusus yang dapat Anda buat atau edit di Excel. File CSV menyimpan informasi yang dipisahkan oleh koma, bukan menyimpan informasi dalam kolom. Saat teks dan angka disimpan dalam file CSV, mudah untuk memindahkannya dari satu program ke program lain. 

Format data CSV pertama kali digunakan pada tahun 1978, CSV baru muncul dan mulai digunakan pada tahun 1983 

Contoh :
\lstinputlisting[firstline=8, lastline=20]{src/4/1174091/teori/j1_1174091.py}

\item Aplikasi apa saja yang bisa menciptakan file csv

Numbers,Google Sheet,SiMBA dan Semua aplikasi teks editor seperti notepad++, vscode, sublime 

\item Cara menulis dan membaca file csv di excel.
\begin{figure}
\begin{itemize}
	\item pertama buat apa yang akan di isi
	\item setelah itu pijit file->save as-> lalu ubah save as type menjadi csv

\end{itemize}
\end{figure}
\item Jelaskan sejarah library csv

Module csv mengimplementasikan kelas untuk membaca dan menulis data kedalam format CSV. Hal ini memungkinkan programmer untuk "tulis data ini dalam format yang disukai oleh Excel," atau "baca data dari file yang dihasilkan oleh Excel," tanpa mengetahui detail yang tepat dari format CSV yang digunakan oleh Excel. Pemrogram juga dapat menggambarkan format CSV yang dipahami oleh aplikasi lain atau menentukan format CSV tujuan khusus untuk mereka sendiri.

\item Jelaskan sejarah library pandas

Pandas merupakan toolkit yang powerfull sebagai alat analisis data dan struktur untuk bahasa pemrograman Python. Dengan menggunakan pandas kita dapat mengolah data dengan mudah, salah satu fiturnya adalah Dataframe.

\item Jelaskan fungsi-fungsi yang terdapat di library csv
\begin{itemize}
	\item csv.reader
	
	membaca sbeuah file CSV yang telah dihasilkan aplikasi atau program lain. 
	
	\item csv.writer
	
	Berfungsi untuk menuliskan data dari variable kedalam file csv.
	
	\item csv.register\textunderscore dialect
	
	Mendaftarkan dialect pada csv
	\item csv.unregister\textunderscore dialect
	
	Menghapus dialect yang telah didaftarkan
	
	\item csv.list\textunderscore dialects
	
	Mengembalikan dialect menjadi list
	
	\item csv.field\textunderscore size\textunderscore limit
	
	Mengembalikan ukuran field maksimum yang diizinkan oleh parser.
	
	\item csv.DictReader
	
	membaca csv file sebagai csv file
	
\end{itemize}

\item Jelaskan fungsi-fungsi yang terdapat di library pandas
\begin{itemize}
	\item pandas.read\textunderscore csv
	
	Berfungsi untuk membaca dan mengembalikan data kedalam format DataFrame. 
	
	\item index\textunderscore col

	Berfungsi untuk mengubah variable menjadi index
	\item to\textunderscore csv
	
	Berfungsi untuk mengedit data dalam csv

\end{itemize}
\end{enumerate}
%%%%%%%%%%%%%%%%%%%%%%%%%%%%%%%%%%%%%%%%%%%%%%%%%%%%%%%%%%%%%%%%%%%%%%%%%%%%%%%%%%%%%%%%%%%%%%%%%%%%%%%	