%\section{Tomy Prawoto}
%\subsection{Soal 1}
%Isi jawaban soal ke-1

%Kalau mau dibikin paragrap \textbf{cukup enter aja}, tidak usah pakai \verb|par| dsb

%\subsection{Soal 2}
%Isi jawaban soal ke-2

%\subsection{Soal 3}
%Isi jawaban soal ke-3

%%%%%%%%%%%%%%%%%%%%%%%%%%%%%%%%%%%%%%%%%%%%%%%%%%%%%%%%%%%%%%%%%%%%%%%%%%%%%%%%%%%%%%%%%%%%%%%%%%%

\section{Kaka Kamaludin}
\subsection{Soal 1}
folder /dev pada linux beriisi file konfigurasi hardware.
Device Manager berfungsi untuk mengatur driver hardware.

\subsection{Soal 2}
\begin{itemize}
	\item download Arduino Software(IDE)  
		https://www.arduino.cc/en/Main/Software
		download untuk linux
	\item extract file yang di download dan masuk ke folder hasil extract
	\item jalankan "./install.sh"
\end{itemize}

\subsection{Soal 3}
. . .

\subsection{Soal 4}
modul pyserial berfungsi untuk merangkum akses untuk port serial. moduk ini dapat di gunakan untuk python yang berjalan pada windows, osx, BSD (yang mendukung system POSIX) dan IronPython.Ini dirilis di bawah lisensi perangkat lunak gratis.

\subsection{Soal 5}
\begin{itemize}
	\item serial.Serial('/dev/tty*')
	membuka port serial
	
	\item ser.close()
	menutup port serial
	
	\item ser.read()
	membaca satu bit
	
	\item ser.readline()
	membaca line semua line
		
\end{itemize}

\subsection{Soal 6}
fungsi perulangan dibutuhkan untuk penggunaan code membutuhkan penggunaan contoh nya seperti multiple choce yang menggunakan perulangan tak terhingga 'while'.

\subsection{Soal 7}
\lstinputlisting[firstline=1, lastline=9]{src/5/1174067/Teori/1174067.py}